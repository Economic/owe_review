\onehalfspacing
\subsection*{Addison Blackburn Cotti 2009}
\vspace{-0.7em}

\noindent John T. Addison, McKinley L. Blackburn and Chad D. Cotti. ``Do Minimum Wages Raise Employment? Evidence from the U.S. Retail-trade Sector'', \href{https://doi.org/10.1016/j.labeco.2008.12.007}{Labour Economics}.

\vspace{0.7em}

\noindent {\bf Own-wage elasticity estimate:} 0.452 (0.436)

\vspace{0.7em}

\noindent {\bf Source of estimate:} Weighted average of wage and employment elasticities for three sectors with statistically positive wage effects: convenience stores; specialty food stores; beer, wine, and liquor stores. Elasticities from Table 4 (estimates with county-level trends). Weights are mean employment counts from Table 1.

\subsection*{Addison Blackburn Cotti 2012}
\vspace{-0.7em}

\noindent John T. Addison, McKinley L. Blackburn and Chad D. Cotti. ``The Effect of Minimum Wages on Labour Market Outcomes: County-Level Estimates from the Restaurant-and-Bar Sector'', \href{https://doi.org/10.1111/j.1467-8543.2010.00819.x}{British Journal of Industrial Relations}.

\vspace{0.7em}

\noindent {\bf Own-wage elasticity estimate:} -0.035 (0.193)

\vspace{0.7em}

\noindent {\bf Source of estimate:} Wage and employment elasticities from Table 3 (estimates with county-level trends). Authors write that "specifications with trends are to be preferred" (p.426).

\subsection*{Addison Blackburn Cotti 2013}
\vspace{-0.7em}

\noindent John T. Addison, McKinley L. Blackburn and Chad D. Cotti. ``Minimum Wage Increases in a Recessionary Environment'', \href{https://doi.org/10.1016/j.labeco.2013.02.004}{Labour Economics}.

\vspace{0.7em}

\noindent {\bf Own-wage elasticity estimate:} -0.084 (0.122)

\vspace{0.7em}

\noindent {\bf Source of estimate:} Earnings and employment elasticities from Table 2, row 1 (food and drinking places), "county trends" specifications. Authors prefer this specification over "basic" and "border county": "our preference is to direclty model county-specific trends, and with this approach the estimates are quite precise" (p.35). No reported wage effects for analysis using CPS or ACS, or for unemployment heterogeneity analysis.

\subsection*{Addison Blackburn Cotti 2015}
\vspace{-0.7em}

\noindent John T. Addison, McKinley L. Blackburn and Chad D. Cotti. ``On the Robustness of Minimum Wage Effects: Geographically-Disparate Trends and Job Growth Equations'', \href{https://doi.org/10.1186/s40172-015-0039-z}{IZA Journal of Labor Economics}.

\vspace{0.7em}

\noindent {\bf Own-wage elasticity estimate:} -0.251 (0.138)

\vspace{0.7em}

\noindent {\bf Source of estimate:} Employment and earnings elasticities for restaurant and bar sector using county-specific linear trends, because the authors select that specification as their "preferred model" on page 7, for the 1990-2014 period, which is the longest period in their sample. Employment elasticity from the top panel of Table 3, column "1st". Earnings elasticity from the top panel of Table 6, column "1st".

\subsection*{Aitken Dolton Riley 2019}
\vspace{-0.7em}

\noindent Andrew Aitken, Peter Dolton and Rebecca Riley. ``The Impact of the Introduction of the National Living Wage on Employment, Hours, and Wages'', \href{https://www.niesr.ac.uk/wp-content/uploads/2021/10/DP501.pdf}{NIESR Discussion Paper}.

\vspace{0.7em}

\noindent {\bf Own-wage elasticity estimate:} -0.254 (0.223)

\vspace{0.7em}

\noindent {\bf Source of estimate:} Percent wage change from Table 5, row 1, specification 1. To calculate the percent employment change, we take the percentage point employment change from Table 7, row 1, specification 1 and divide it by the 0.67 treated group mean in 2015. These are the "2015 as treatment year" estimates, estimating the effect of the 2016 increase, as opposed to the "2016 as treatment year" estimates, which focus on the 2017 uprating.

\subsection*{Allegretto Dube Reich 2011}
\vspace{-0.7em}

\noindent Sylvia Allegretto, Arindrajit Dube and Michael Reich. ``Do Minimum Wages Really Reduce Teen Employment? Accounting for Heterogeneity and Selectivity in State Panel Data'', \href{https://doi.org/10.1111/j.1468-232X.2011.00634.x}{Industrial Relations}.

\vspace{0.7em}

\noindent {\bf Own-wage elasticity estimate:} 0.315 (0.402)

\vspace{0.7em}

\noindent {\bf Source of estimate:} Wage and employment elasticities from Table 3, specification 4, row "All teens". Employment elasticity standard error calculated by scaling standard error of semi-elasticity by ratio of elasticity to semi-elasticity.

\subsection*{Allegretto Dube Reich Zipperer 2017}
\vspace{-0.7em}

\noindent Sylvia Allegretto, Arindrajit Dube, Michael Reich and Ben Zipperer. ``Credible Research Designs for Minimum Wage Studies'', \href{https://doi.org/10.1177/0019793917692788}{ILR Review}.

\vspace{0.7em}

\noindent {\bf Own-wage elasticity estimate:} 0.043 (0.188)

\vspace{0.7em}

\noindent {\bf Source of estimate:} Teen wage and employment elasticities from Table 1, specification 2 (linear trends), row "Division-period FE".

\subsection*{Allegretto and Nadler 2015}
\vspace{-0.7em}

\noindent Sylvia Allegretto and Carl Nadler. ``Tipped Wage Effects on Earnings and Employment in Full-Service Restaurants'', \href{https://doi.org/10.1111/irel.12108}{Industrial Relations}.

\vspace{0.7em}

\noindent {\bf Own-wage elasticity estimate:} -0.024 (0.662)

\vspace{0.7em}

\noindent {\bf Source of estimate:} Sum of the headline and tipped wage and employment elasticities to approximate the effect of a marginal increase in both the headline and tipped minimum wage. Wage and employment elasticities from Table 2, Panel A (Full service Restaurants), specification 4 (Division-period effects and state-specific time trends), adding the tipped and headline minimum effects together. Standard errors assume no covariance between the tipped and headline minimum wage effect estimates.

\subsection*{Azar Huet-Vaughn Marinescu Taska Wachter 2023}
\vspace{-0.7em}

\noindent José Azar, Emiliano Huet-Vaughn, Ioana Elena Marinescu, Bledi Taska and Till Von Wachter. ``Minimum Wage Employment Effects and Labor Market Concentration'', \href{https://doi.org/10.1093/restud/rdad091}{The Review of Economic Studies}.

\vspace{0.7em}

\noindent {\bf Own-wage elasticity estimate:} 0.088 (0.925)

\vspace{0.7em}

\noindent {\bf Source of estimate:} Weighted average of elasticities and their variances for the three mid-HHI occupation OWEs presented in Figure 7 (Stock Clerks, Retail Sales, and Cashiers), where the weights are the sample sizes from Table 3. Authors emailed the three OWEs coefficients and standard errors used in Figure 7.

\subsection*{Bailey DiNardo Stuart 2021}
\vspace{-0.7em}

\noindent Martha J. Bailey, John DiNardo and Bryan A. Stuart. ``The Economic Impact of a High National Minimum Wage: Evidence from the 1966 Fair Labor Standards Act'', \href{https://doi.org/10.1086/712554}{Journal of Labor Economics}.

\vspace{0.7em}

\noindent {\bf Own-wage elasticity estimate:} -0.135 (0.077)

\vspace{0.7em}

\noindent {\bf Source of estimate:} Own-wage elasticity reported in Table 3, panel A (Employed during year), specification 3. Selected this particular estimate with authors over email.

\subsection*{Baskaya and Rubinstein 2015}
\vspace{-0.7em}

\noindent Yusuf Soner Baskaya and Yona Rubinstein. ``Using Federal Minimum Wages to Identify the Impact of Minimum Wages on Employment and Earnings across the U.S. States'', \href{NA}{Working Paper}.

\vspace{0.7em}

\noindent {\bf Own-wage elasticity estimate:} -0.899 (0.609)

\vspace{0.7em}

\noindent {\bf Source of estimate:} Employment elasiticity is the average of estimates from Table 10a, Panel A, row 1, specifications 5 and 8, where the estimates are divided by the sample mean employment rate in Appendix Table A2, column 1, last row. Wage elasticity is the average of estimates from Table 10b, Panel A, row 1, specifications 5 and 8.

\subsection*{Berge and Frings 2019}
\vspace{-0.7em}

\noindent Philipp vom Berge and Hanna Frings. ``High-impact minimum wages and heterogeneous regions'', \href{https://doi.org/10.1007/s00181-019-01661-0}{Empirical Economics}.

\vspace{0.7em}

\noindent {\bf Own-wage elasticity estimate:} -1.567 (1.923)

\vspace{0.7em}

\noindent {\bf Source of estimate:} Wage and employment effects are the average across three specifications. Wage growth effects from Table 1, row "Treatment effect (East)", specifications 3 through 5. Employment growth effects from Table 3, row "Treatment effect (East)", specifications 3 through 5.

\subsection*{Bezooijen Berge Salomons 2024}
\vspace{-0.7em}

\noindent Emiel van Bezooijen, Wiljan van den Berge and Anna Salomons. ``The Young Bunch: Youth Minimum Wages and Labor Market Outcomes'', \href{https://doi.org/10.1177/00197939241239317}{ILR Review}.

\vspace{0.7em}

\noindent {\bf Own-wage elasticity estimate:} 0.226 (0.143)

\vspace{0.7em}

\noindent {\bf Source of estimate:} Own-wage elasticity from Table 4, Jobs specification

\subsection*{Borgshulte and Cho 2020}
\vspace{-0.7em}

\noindent Mark Borgshulte and Heepyung Cho. ``Minimum Wages and Retirement'', \href{https://doi.org/10.1177/0019793919845861}{ILR Review}.

\vspace{0.7em}

\noindent {\bf Own-wage elasticity estimate:} 0.419 (0.301)

\vspace{0.7em}

\noindent {\bf Source of estimate:} The statistically significant earnings elasticity estimate is from Table 3, specification 5, Panel E. The analogous employment elasticity estimate is Table 3, specification 5, panel A.

\subsection*{Bossler and Gerner 2020}
\vspace{-0.7em}

\noindent Mario Bossler and Hans-Dieter Gerner. ``Employment Effects of the new German Minimum Wage: Evidence from Establishment-level Microdata'', \href{https://doi.org/10.1177/0019793919889635}{ILR Review}.

\vspace{0.7em}

\noindent {\bf Own-wage elasticity estimate:} -0.278 (0.212)

\vspace{0.7em}

\noindent {\bf Source of estimate:} Own-wage elasticity reported in Table 2, specification 3 (Derived labor demand elasticity), Panel B (Intensive treatment definition). Authors emailed to select this specification.

\subsection*{Brown and Herbst 2023}
\vspace{-0.7em}

\noindent Jessica H. Brown and Chris M. Herbst. ``Minimum Wage, Worker Quality, and Consumer Well-Being: Evidence from the Child Care Market'', \href{https://docs.iza.org/dp16257.pdf}{IZA Discussion Paper}.

\vspace{0.7em}

\noindent {\bf Own-wage elasticity estimate:} 0.841 (0.855)

\vspace{0.7em}

\noindent {\bf Source of estimate:} Earnings and employment elasticities reported in specification 1 (All) of Table 1.

\subsection*{Burkhauser Couch Wittenburg 2000}
\vspace{-0.7em}

\noindent Richard V. Burkhauser, Kenneth A. Couch and David C. Wittenburg. ``A Reassessment of the New Economics of the Minimum Wage Literature with Monthly Data from the Current Population Survey'', \href{https://doi.org/10.1086/209972}{Journal of Labor Economics}.

\vspace{0.7em}

\noindent {\bf Own-wage elasticity estimate:} -0.604 (0.184)

\vspace{0.7em}

\noindent {\bf Source of estimate:} Wage elasticity from Table 2, specification 1 (1979-1992, seasonal, state, and year effects). Employment elasticity from the "controls for state-level autocorrelation and heteroskedasticity correction" elasticity reported in Table 7, specification Table 3 (4) (1979-1992, seasonal, state, and year effects), where the semi-elasticity standard error is scaled by the ratio of the elasticity to semi-elasticity. Authors prefer models without year effects but our review is restricted to DD-style estimators as opposed to repeated time series, so we chose the year fixed effect estimate in paper that had statistically significant wage effects and a similar employment effect specification that seem to be preferred by author.

\subsection*{Butcher and Dickens 2023}
\vspace{-0.7em}

\noindent Tim Butcher and Richard Dickens. ``Impact of the National Living Wage using Geographic, Age and Gender Wage Variation'', \href{https://www.gov.uk/government/publications/national-living-wage-impacts-by-geography-age-and-gender}{Low Pay Commission}.

\vspace{0.7em}

\noindent {\bf Own-wage elasticity estimate:} 0.476 (0.395)

\vspace{0.7em}

\noindent {\bf Source of estimate:} Wage percent change from Table 3, column 4. Employment percent change is the percentage point change in employment from Table 4, column 4, divided by the overall employment rate of 0.65 (see footnote 65).

\subsection*{Campolieti Gunderson Riddell 2006}
\vspace{-0.7em}

\noindent Michele Campolieti, Morley Gunderson and Chris Riddell. ``Minimum Wage Impacts from a Prespecified Research Design: Canada 1981-1997'', \href{https://doi.org/10.1111/j.1468-232X.2006.00424.x}{Industrial Relations}.

\vspace{0.7em}

\noindent {\bf Own-wage elasticity estimate:} -0.548 (0.359)

\vspace{0.7em}

\noindent {\bf Source of estimate:} Own-wage elasticity reported in Table 4, panel "Including prime-age skilled employment rate", specification "Youths 16-24 combined".

\subsection*{Card 1992a}
\vspace{-0.7em}

\noindent David Card. ``Do Minimum Wages Reduce Employment? A Case Study of California, 1987-89'', \href{https://doi.org/10.1177/001979399204600104}{ILR Review}.

\vspace{0.7em}

\noindent {\bf Own-wage elasticity estimate:} 1.333 (0.609)

\vspace{0.7em}

\noindent {\bf Source of estimate:} Estimates from "Difference in Differences" column of Table 4. Wage elasticity from row "Mean Log Wage". Employment semi-elasticity and standard error from row "Employment rate (%)" are divided by 42, the mean employment rate reported in the column California Teens, 1987.

\subsection*{Card 1992b}
\vspace{-0.7em}

\noindent David Card. ``Using Regional Variation in Wages to Measure the Effects of the Federal Minimum Wage'', \href{https://doi.org/10.1177/001979399204600103}{ILR Review}.

\vspace{0.7em}

\noindent {\bf Own-wage elasticity estimate:} 0.352 (0.616)

\vspace{0.7em}

\noindent {\bf Source of estimate:} Own-wage semi-elasticity reported in Table 4, specification 5, scaled by the mean of teen employment rates in 1989 across all groups (0.455).

\subsection*{Card Katz Krueger 1994}
\vspace{-0.7em}

\noindent David Card, Lawrence F. Katz and Alan B. Krueger. ``Comment on David Neumark and William Wascher, "Employment Effects of Minimum and Subminimum Wages: Panel Data on State Minimum Wage Laws"'', \href{https://doi.org/10.1177/001979399404700308}{ILR Review}.

\vspace{0.7em}

\noindent {\bf Own-wage elasticity estimate:} -0.043 (0.15)

\vspace{0.7em}

\noindent {\bf Source of estimate:} Own-wage semi-elasticity reported in Table 2, row 5, specification 2, divided by unweighted teen epop mean in footnote b (0.466).

\subsection*{Cengiz Dube Lindner Zentler-Munro 2022}
\vspace{-0.7em}

\noindent Doruk Cengiz, Arindrajit Dube, Attila Lindner and David Zentler-Munro. ``Seeing beyond the Trees: Using Machine Learning to Estimate the Impact of Minimum Wages on Labor Market Outcomes'', \href{https://doi.org/10.1086/718497}{Journal of Labor Economics}.

\vspace{0.7em}

\noindent {\bf Own-wage elasticity estimate:} 0.114 (0.216)

\vspace{0.7em}

\noindent {\bf Source of estimate:} Own-wage elasticity reported in Table 2, specification 3 , row "Employment elasticity with respect to wage" (High recall, Boosted tree)

\subsection*{Cengiz Dube Lindner Zipperer 2019}
\vspace{-0.7em}

\noindent Doruk Cengiz, Arindrajit Dube, Attila Lindner and Ben Zipperer. ``The Effect of Minimum Wages on Low-Wage Jobs'', \href{https://doi.org/10.1093/qje/qjz014}{Quarterly Journal of Economics}.

\vspace{0.7em}

\noindent {\bf Own-wage elasticity estimate:} 0.411 (0.43)

\vspace{0.7em}

\noindent {\bf Source of estimate:} Own-wage elasticity reported in Table 1, specification 1, row "Emp. elasticity w.r.t affected wage"

\subsection*{Clemens and Strain 2021}
\vspace{-0.7em}

\noindent Jeffrey Clemens and Michael R. Strain. ``The Heterogeneous Effects of Large and Small Minimum Wage Changes: Evidence over the Short and Medium Run Using a Pre-Analysis Plan'', \href{https://www.nber.org/system/files/working_papers/w29264/w29264.pdf}{NBER Working Paper}.

\vspace{0.7em}

\noindent {\bf Own-wage elasticity estimate:} -0.245 (NA)

\vspace{0.7em}

\noindent {\bf Source of estimate:} Authors suggested simple average of own-wage elasticities reported in Table 10, specifications 1 (Low-Skill, All Changers) and 5 (Young, All Changers).

\subsection*{Coviello Deserranno Persico 2022}
\vspace{-0.7em}

\noindent Decio Coviello, Erika Deserranno and Nicola Persico. ``Minimum Wage and Individual Worker Productivity: Evidence from a Large US Retailer'', \href{https://doi.org/10.1086/720397}{Journal of Political Economy}.

\vspace{0.7em}

\noindent {\bf Own-wage elasticity estimate:} -0.219 (NA)

\vspace{0.7em}

\noindent {\bf Source of estimate:} Own-wage elasticity reported on p. 2352 and footnote 43.

\subsection*{Currie and Fallick 1996}
\vspace{-0.7em}

\noindent Janet Currie and Bruce C. Fallick. ``The Minimum Wage and the Employment of Youth Evidence from the NLSY'', \href{https://doi.org/10.2307/146069}{The Journal of Human Resources}.

\vspace{0.7em}

\noindent {\bf Own-wage elasticity estimate:} -0.891 (0.327)

\vspace{0.7em}

\noindent {\bf Source of estimate:} Employment elasticity uses semi-elasticity from table 2, specification 4 (Fixed Effects), divided by 0.97, the proportion employed in next year, 1987, from Table 1. Wage elasticity from Table 4, Panel B, specification 2 (Fixed Effects).

\subsection*{Derenoncourt and Montialoux 2021}
\vspace{-0.7em}

\noindent Ellora Derenoncourt and Claire Montialoux. ``Minimum Wages and Racial Inequality'', \href{https://doi.org/10.1093/qje/qjaa031}{Quarterly Journal of Economics}.

\vspace{0.7em}

\noindent {\bf Own-wage elasticity estimate:} 0.06 (0.16)

\vspace{0.7em}

\noindent {\bf Source of estimate:} Own-wage elasticity reported in Table 6, first column (All) of "Baseline cross-state design" specification, row "Emp. vs (unemp/nilf) elast.". This row is referred to in authors' comparison of own-wage elasticities across specifications in Section V.A.5.

\subsection*{Dickens Riley Wilkinson 2015}
\vspace{-0.7em}

\noindent Richard Dickens, Rebecca Riley and David Wilkinson. ``A Re-examination of the Impact of the UK National Minimum Wage on Employmen'', \href{https://doi.org/10.1111/ecca.12158}{Economica}.

\vspace{0.7em}

\noindent {\bf Own-wage elasticity estimate:} -0.596 (0.482)

\vspace{0.7em}

\noindent {\bf Source of estimate:} Weighted average of wage and employment elasticities across three groups: female full-time, female part-time, and male full-time. Weights from Table 2 pre-treatment observation counts. Wage elasticities from Table 3, line 1998. Employment semi-elasticity from Table 5, line 1998, scaled by pre-period treatment group-specific employment inferred from Figure 2.

\subsection*{Dow Godøy Lowenstein Reich 2020}
\vspace{-0.7em}

\noindent William H. Dow, Anna Godøy, Christopher A. Lowenstein and Michael Reich. ``Can Economic Policies Reduce Deaths of Despair?'', \href{https://doi.org/10.1016/j.jhealeco.2020.102372}{Journal of Health Economics}.

\vspace{0.7em}

\noindent {\bf Own-wage elasticity estimate:} 0.733 (0.823)

\vspace{0.7em}

\noindent {\bf Source of estimate:} Wage elasticity reported in Table A1, Panel B, specification 1. Employment semi-elasticity reported in Table A1, Panel B, specification 2, divided by mean employment of the regression sample, 0.565, calculated using a modified version of the authors' replication package file wage\_ag051120.do, available here: https://gist.github.com/benzipperer/7860f1055036b95d9e04a212e2b61bed

\subsection*{Draca Machin Reenan 2011}
\vspace{-0.7em}

\noindent Mirko Draca, Stephen Machin and John Van Reenan. ``Minimum Wages and Firm Profitability'', \href{https://doi.org/10.1257/app.3.1.129}{American Economic Journal: Applied Economics}.

\vspace{0.7em}

\noindent {\bf Own-wage elasticity estimate:} 0.484 (1.401)

\vspace{0.7em}

\noindent {\bf Source of estimate:} Simple average from Table 5 of the high-market power and low-market power wage (panel A) and employment elasticities (panel C).

\subsection*{Dube 2019}
\vspace{-0.7em}

\noindent Arindrajit Dube. ``Impacts of minimum wages: review of the international evidence'', \href{https://assets.publishing.service.gov.uk/government/uploads/system/uploads/attachment_data/file/844350/impacts_of_minimum_wages_review_of_the_international_evidence_Arindrajit_Dube_web.pdf}{Her Majesty's Treasury}.

\vspace{0.7em}

\noindent {\bf Own-wage elasticity estimate:} 0.32 (0.37)

\vspace{0.7em}

\noindent {\bf Source of estimate:} Own-wage elasticity reported in text at top of page 33.

\subsection*{Dube Lester Reich 2010}
\vspace{-0.7em}

\noindent Arindrajit Dube, T. William Lester and Michael Reich. ``Minimum Wage Effects Across State Borders: Estimates Using Contiguous Counties'', \href{https://doi.org/10.1162/REST_a_00039}{The Review of Economics and Statistics}.

\vspace{0.7em}

\noindent {\bf Own-wage elasticity estimate:} 0.079 (0.286)

\vspace{0.7em}

\noindent {\bf Source of estimate:} Own-wage elasticity reported in Table 2, specification 6, row "Labor demand elasticity".

\subsection*{Dube Lester Reich 2016}
\vspace{-0.7em}

\noindent Arindrajit Dube, T. William Lester and Michael Reich. ``Minimum Wage Shocks, Employment Flows, and Labor Market Frictions'', \href{https://doi.org/10.1086/685449}{Journal of Labor Economics}.

\vspace{0.7em}

\noindent {\bf Own-wage elasticity estimate:} -0.266 (0.383)

\vspace{0.7em}

\noindent {\bf Source of estimate:} Wage and employment elasticities from Table 3, specification 2. Authors communicated that teens is the preferred group.

\subsection*{Dube Naidu Reich 2007}
\vspace{-0.7em}

\noindent Arindrajit Dube, Suresh Naidu and Michael Reich. ``The Economic Effects of a Citywide Minimum Wage'', \href{https://doi.org/10.1177/001979390706000404}{ILR Review}.

\vspace{0.7em}

\noindent {\bf Own-wage elasticity estimate:} 0.296 (0.622)

\vspace{0.7em}

\noindent {\bf Source of estimate:} Employment log change from Table 7, panel "Ln(employment)", specification 1 ("Full sample"). Wage effect from Table 2, specification 1 ("Full sample"), divided by 10.22, the Table 1, Wave 1, Treatment Group, Average Wage.

\subsection*{Dube and Lindner 2021}
\vspace{-0.7em}

\noindent Arindrajit Dube and Attila Lindner. ``City Limits: What Do Local-Area Minimum Wages Do?'', \href{https://doi.org/10.1257/jep.35.1.27}{Journal of Economic Perspectives}.

\vspace{0.7em}

\noindent {\bf Own-wage elasticity estimate:} -0.12 (0.38)

\vspace{0.7em}

\noindent {\bf Source of estimate:} Own-wage elasticity reported in Figure 3, Panel B.

\subsection*{Dube and Zipperer 2015}
\vspace{-0.7em}

\noindent Arindrajit Dube and Ben Zipperer. ``Pooling Multiple Case Studies Using Synthetic Controls: An Application to Minimum Wage Policies'', \href{https://docs.iza.org/dp8944.pdf}{IZA Discussion Paper}.

\vspace{0.7em}

\noindent {\bf Own-wage elasticity estimate:} -0.135 (0.249)

\vspace{0.7em}

\noindent {\bf Source of estimate:} Employment and wage elasticities from Table 6, Hodges-Lehmann estimate, with standard errors interpolated from the confidence interval.

\subsection*{Dustmann Lindner Schönberg Umkehrer Berge 2022}
\vspace{-0.7em}

\noindent Christian Dustmann, Attila Lindner, Uta Schönberg, Matthias Umkehrer and Philipp vom Berge. ``Reallocation Effects of the Minimum Wage'', \href{https://doi.org/10.1093/qje/qjab028}{Quarterly Journal of Economics}.

\vspace{0.7em}

\noindent {\bf Own-wage elasticity estimate:} 0.03 (0.12)

\vspace{0.7em}

\noindent {\bf Source of estimate:} Own-wage elasticity reported in footnote 27 on page 316.

\subsection*{Eriksson and Pytlikova 2004}
\vspace{-0.7em}

\noindent Tor Eriksson and Mariola Pytlikova. ``Firm-level Consequences of Large Minimum-wage Increases in the Czech and Slovak Republics'', \href{https://doi.org/10.1111/j.1121-7081.2004.00259.x}{LABOUR}.

\vspace{0.7em}

\noindent {\bf Own-wage elasticity estimate:} 0.024 (0.152)

\vspace{0.7em}

\noindent {\bf Source of estimate:} Weighted average of the reported own-wage elasticities in Tables 6 and 7. Czec OWE is the simple average of four OWE estimates in columns three and four of Table 6, Panel 1 (two-thirds) Total and Panel 2 (wage gap). Slovak OWE is the simple average of four OWE estimates in columns three and four of Table 7, Panel 1 (two-thirds) Total and Panel 2 (wage gap). We then report a weighted average of these two country-specific OWEs, where the weights are the total 1998 employment sizes reported in Table 4 (Czech) and Table 5 (Slovak).

\subsection*{Even and Macpherson 2014}
\vspace{-0.7em}

\noindent William E. Even and David A. Macpherson. ``The Effect of the Tipped Minimum Wage on Employees in the U.S. Restaurant Industry'', \href{https://doi.org/10.4284/0038-4038-2012.283}{Southern Economic Journal}.

\vspace{0.7em}

\noindent {\bf Own-wage elasticity estimate:} -0.782 (0.288)

\vspace{0.7em}

\noindent {\bf Source of estimate:} Sum of the headline and tipped wage and employment elasticities to approximate the effect of a marginal increase in both the headline and tipped minimum wage. Wage elasticities from Table 1 and employment elasticities from Table 2, using specification 1 (Full service, 1990:1 to 2011:4, no state-specific time trends), rows "Log of Tipped Minimum Wage" and "Log of Minimum Wage". Standard errors assume no covariance between the tipped and headline minimum wage effect estimates. Standard errors calculated from t-statistics assuming normal distribution.

\subsection*{Gittings and Schmutte 2016}
\vspace{-0.7em}

\noindent R. Kaj Gittings and Ian M. Schmutte. ``Getting Handcuffs on an Octopus: Minimum Wages, Employment, and Turnover'', \href{https://doi.org/10.1177/0019793915623519}{ILR Review}.

\vspace{0.7em}

\noindent {\bf Own-wage elasticity estimate:} 0 (0.75)

\vspace{0.7em}

\noindent {\bf Source of estimate:} Earnings elasticity from Table 4, Panel A, specification "Log (earnings)". Employment semi-elasticity from Table 4, Panel A, specification "Employment / Population", divided by 0.28, the mean QWI Teen end-of-quarter EPOP from Table 1.

\subsection*{Giuliano 2013}
\vspace{-0.7em}

\noindent Laura Giuliano. ``Minimum Wage Effects on Employment, Substitution, and the Teenage Labor Supply: Evidence from Personnel Data'', \href{https://doi.org/10.1086/666921}{Journal of Labor Economics}.

\vspace{0.7em}

\noindent {\bf Own-wage elasticity estimate:} -0.59 (0.613)

\vspace{0.7em}

\noindent {\bf Source of estimate:} Own-wage elasticity reported in Table 4, column 6. To calculate the standard error, we use the reported employment and wage elasticities and their standard errors. Wage elasticity from Table 4, column 6, row 1. Employment semi elasticity from Table 4, row 2, column 6 divided by the Table 1, All Stores FTE mean of 14.7.

\subsection*{Giupponi Joyce Lindner Waters Wernham Xu 2024}
\vspace{-0.7em}

\noindent Giulia Giupponi, Robert Joyce, Attila Lindner, Tom Waters, Thomas Wernham and Xiaowei Xu. ``The Employment and Distributional Impacts of Nationwide Minimum Wage Changes'', \href{https://doi.org/10.1086/728471}{Journal of Labor Economics}.

\vspace{0.7em}

\noindent {\bf Own-wage elasticity estimate:} -0.2 (0.32)

\vspace{0.7em}

\noindent {\bf Source of estimate:} Own-wage elasticity reported in Table 2, Panel A.

\subsection*{Godoey and Reich 2021}
\vspace{-0.7em}

\noindent Anna Godoey and Michael Reich. ``Are Minimum Wage Effects Greater in Low-Wage Areas?'', \href{https://doi.org/10.1111/irel.12267}{Industrial Relations}.

\vspace{0.7em}

\noindent {\bf Own-wage elasticity estimate:} 0.124 (NA)

\vspace{0.7em}

\noindent {\bf Source of estimate:} Own-wage elasticity reported in Table A2, sample: High school or less, specification 1 (All).

\subsection*{Godøy Reich Wursten Allegretto 2024}
\vspace{-0.7em}

\noindent Anna Godøy, Michael Reich, Jesse Wursten and Sylvia Allegretto. ``Parental Labor Supply: Evidence from Minimum Wage Changes'', \href{https://doi.org/10.3368/jhr.1119-10540R2}{The Journal of Human Resources}.

\vspace{0.7em}

\noindent {\bf Own-wage elasticity estimate:} 0.09 (0.41)

\vspace{0.7em}

\noindent {\bf Source of estimate:} Own-wage elasticity reported in Table 2, column "Any, Any".

\subsection*{Gopalan Hamilton Kalda Sovich 2021}
\vspace{-0.7em}

\noindent Radhakrishnan Gopalan, Barton H. Hamilton, Ankit Kalda and David Sovich. ``State Minimum Wages, Employment, and Wage Spillovers: Evidence from Administrative Payroll Data'', \href{https://doi.org/10.1086/711355}{Journal of Labor Economics}.

\vspace{0.7em}

\noindent {\bf Own-wage elasticity estimate:} -0.38 (NA)

\vspace{0.7em}

\noindent {\bf Source of estimate:} Own-wage elasticity reported on page 696 for total employment ("implied total labor demand elasticity"). Other own-wage elasticities are reported but this seems to be the broadest.

\subsection*{Hampton and Totty 2023}
\vspace{-0.7em}

\noindent Matt Hampton and Evan Totty. ``Minimum wages, retirement timing, and labor supply'', \href{https://doi.org/10.1016/j.jpubeco.2023.104924}{Journal of Public Economics}.

\vspace{0.7em}

\noindent {\bf Own-wage elasticity estimate:} 0.072 (0.134)

\vspace{0.7em}

\noindent {\bf Source of estimate:} Weighted average of the own-wage elasticities reported in Table 7, where the weights are the sample sizes in Panel A. Standard errors for each group are calculated from the confidence interval span divided by 2 times 1.96.

\subsection*{Harasztosi and Lindner 2019}
\vspace{-0.7em}

\noindent Peter Harasztosi and Attila Lindner. ``Who Pays for the Minimum Wage'', \href{https://doi.org/10.1257/aer.20171445}{American Economic Review}.

\vspace{0.7em}

\noindent {\bf Own-wage elasticity estimate:} -0.18 (0.03)

\vspace{0.7em}

\noindent {\bf Source of estimate:} Own-wage elasticity reported by authors on page 2695.

\subsection*{Hirsch Kaufman Zelenska 2015}
\vspace{-0.7em}

\noindent Barry T. Hirsch, Bruce E. Kaufman and Tetyana Zelenska. ``Minimum Wage Channels of Adjustment'', \href{https://doi.org/10.1111/irel.12091}{Industrial Relations}.

\vspace{0.7em}

\noindent {\bf Own-wage elasticity estimate:} 0.392 (0.328)

\vspace{0.7em}

\noindent {\bf Source of estimate:} Own-wage elasticity reported in Table 5, specification 7 (IV, Store FE), which authors refer to as their "preferred employment elasticity" on page 218.

\subsection*{Huet-Vaughn and Piqueras 2023}
\vspace{-0.7em}

\noindent Emiliano Huet-Vaughn and Jon Piqueras. ``The Asymmetric Effect of Wage Floors: A Natural Experiment with a Rising and Falling Minimum Wage'', \href{https://docs.iza.org/dp16684.pdf}{IZA Discussion Paper}.

\vspace{0.7em}

\noindent {\bf Own-wage elasticity estimate:} -1.864 (NA)

\vspace{0.7em}

\noindent {\bf Source of estimate:} Wage change from Figure 2 inset and employment change from Figure 3 inset.

\subsection*{Jardim Long Plotnick Inwegen Vigdor Wething 2022}
\vspace{-0.7em}

\noindent Ekaterina Jardim, Mark C. Long, Robert Plotnick, Emma van Inwegen, Jacob Vigdor and Hilary Wething. ``Minimum-Wage Increases and Low-Wage Employment: Evidence from Seattle'', \href{https://doi.org/10.1257/pol.20180578}{American Economic Journal: Economic Policy}.

\vspace{0.7em}

\noindent {\bf Own-wage elasticity estimate:} -1.75 (NA)

\vspace{0.7em}

\noindent {\bf Source of estimate:} Wage elasticity from Table 6a, specification "SC levels", row "2016:III". Employment elasticity from Table 6c, specification "SC levels", row "2016:III". Communication with authors suggests synthetic control "levels" specification.

\subsection*{Jha Neumark Rodriguez-Lopez 2024}
\vspace{-0.7em}

\noindent Priyaranjan Jha, David Neumark and Antonio Rodriguez-Lopez. ``What's across the Border? Re-Evaluating the Cross-Border Evidence on Minimum Wage Effects'', \href{https://sites.socsci.uci.edu/~jantonio/Papers/minwage_czones.pdf}{Journal of Political Economy Microeconomics}.

\vspace{0.7em}

\noindent {\bf Own-wage elasticity estimate:} -1.485 (0.89)

\vspace{0.7em}

\noindent {\bf Source of estimate:} Wage and employment elasticities from Table 3, specification 1.

\subsection*{Karabarbounis Lise Nath 2023}
\vspace{-0.7em}

\noindent Loukas Karabarbounis, Jeremy Lise and Anusha Nath. ``Minimum Wages and Labor Markets in the Twin Cities'', \href{https://www.nber.org/system/files/working_papers/w30239/w30239.pdf}{NBER Working Paper}.

\vspace{0.7em}

\noindent {\bf Own-wage elasticity estimate:} -1.103 (NA)

\vspace{0.7em}

\noindent {\bf Source of estimate:} Simple average of Minneapolis and St. Paul elasticities from Table 2, Baseline panel, 2021 values. Wage elasticities from "Wage" columns and employment elasticities from "Jobs" columns. Authors refer to Table 2 estimates in a discussion of own-wage elasticities in Section 5.3.

\subsection*{Katz and Krueger 1992}
\vspace{-0.7em}

\noindent Lawrence F. Katz and Alan B. Krueger. ``The Effect of Minimum Wages on the Fast-Food Industry'', \href{https://doi.org/10.1177/001979399204600102}{ILR Review}.

\vspace{0.7em}

\noindent {\bf Own-wage elasticity estimate:} 1.734 (0.934)

\vspace{0.7em}

\noindent {\bf Source of estimate:} Own-wage elasticity reported in Table 5, specification 2.

\subsection*{Kim and Taylor 1995}
\vspace{-0.7em}

\noindent Taeil Kim and Lowell J. Taylor. ``The Employment Effect in Retail Trade of California’s 1988 Minimum Wage Increase'', \href{https://doi.org/10.1080/07350015.1995.10524591}{Journal of Business and Economic Statistics}.

\vspace{0.7em}

\noindent {\bf Own-wage elasticity estimate:} -0.879 (0.133)

\vspace{0.7em}

\noindent {\bf Source of estimate:} Own-wage elasticity reported in Table 4, specification 8 (IV).

\subsection*{Leung 2021}
\vspace{-0.7em}

\noindent Justin H. Leung. ``Minimum Wage and Real Wage Inequality: Evidence from Pass-Through to Retail Prices'', \href{https://doi.org/10.1162/rest_a_00915}{The Review of Economics and Statistics}.

\vspace{0.7em}

\noindent {\bf Own-wage elasticity estimate:} 0.034 (0.654)

\vspace{0.7em}

\noindent {\bf Source of estimate:} Earnings and employment elasticities from Table 4, specification 2 (Grocery). Paper includes results for other sectors, but the main focus appears to be grocery stores where price effects are clear.

\subsection*{Liu Hyclack Regmi 2016}
\vspace{-0.7em}

\noindent Shanshan Liu, Thomas J. Hyclack and Krishna Regmi. ``Impact of the Minimum Wage on Youth Labor Markets'', \href{https://doi.org/10.1111/labr.12071}{LABOUR}.

\vspace{0.7em}

\noindent {\bf Own-wage elasticity estimate:} -0.828 (0.254)

\vspace{0.7em}

\noindent {\bf Source of estimate:} Employment and wage elasticities from Table 2, specifications "14-18", Panel B.

\subsection*{Machin Manning Rahman 2003}
\vspace{-0.7em}

\noindent Stephen Machin, Alan Manning and Lupin Rahman. ``Where the Minimum Wage Bites Hard: Introduction of Minimum Wages to a Low Wage Sector'', \href{https://doi.org/10.1162/154247603322256792}{Journal of the European Economic Association}.

\vspace{0.7em}

\noindent {\bf Own-wage elasticity estimate:} -0.453 (0.251)

\vspace{0.7em}

\noindent {\bf Source of estimate:} Simple average of own-wage elasticities reported in Table 6, specifications 7 and 8, panel "Change in log number employed".

\subsection*{Machin and Wilson 2004}
\vspace{-0.7em}

\noindent Stephen Machin and Joan Wilson. ``Minimum Wages in a Low-Wage Labour Market: Care Homes in the UK'', \href{https://doi.org/10.1111/j.0013-0133.2003.00199.x}{The Economic Journal}.

\vspace{0.7em}

\noindent {\bf Own-wage elasticity estimate:} -0.467 (0.077)

\vspace{0.7em}

\noindent {\bf Source of estimate:} CONFIRM SOURCE TK

\subsection*{Manning 2021}
\vspace{-0.7em}

\noindent Alan Manning. ``The Elusive Employment Effect of the Minimum Wage'', \href{https://doi.org/10.1257/jep.35.1.3}{Journal of Economic Perspectives}.

\vspace{0.7em}

\noindent {\bf Own-wage elasticity estimate:} 0.011 (0.404)

\vspace{0.7em}

\noindent {\bf Source of estimate:} Wage and employment elasticities are the hours-weighted average of unweighted average elasticities (across seven specifications) for ages 16-19 and ages 20-24. Group Age 16-19 receives a weight of about 29% and group Age 20-24 receives a weight of 71%, according to the average "Percent of total hours" worked across 1979-2019, reported in Table 1. Wage elasticities are from Table A1, Panel A and Panel B. Employment elasticities are from Table A2, Panel A and Panel B.

\subsection*{Monras 2019}
\vspace{-0.7em}

\noindent Joan Monras. ``Minimum Wages and Spatial Equilibrium: Theory and Evidence'', \href{https://doi.org/10.1086/702650}{Journal of Labor Economics}.

\vspace{0.7em}

\noindent {\bf Own-wage elasticity estimate:} -0.667 (NA)

\vspace{0.7em}

\noindent {\bf Source of estimate:} Own-wage elasticity reported in Table 3, specification "Model 4", row "Implied local labor demand elasticity, FTE". Specification confirmed with author by email.

\subsection*{Nadler Allegretto Godoey Reich 2019}
\vspace{-0.7em}

\noindent Carl Nadler, Sylvia Allegretto, Anna Godoey and Michael Reich. ``Are Local Minimum Wages Too High and How Could We Even Know?'', \href{http://irle.berkeley.edu/files/2019/04/Are-Local-Minimum-Wages-Too-High.pdf}{IRLE Working Paper}.

\vspace{0.7em}

\noindent {\bf Own-wage elasticity estimate:} 0.21 (0.201)

\vspace{0.7em}

\noindent {\bf Source of estimate:} Earnings elasticity from Table 3, specification 2, panel B. Employment elasticity from Table 3, specification 4, panel B.

\subsection*{Neumark Schweitzer Wascher 2004}
\vspace{-0.7em}

\noindent David Neumark, Mark Schweitzer and William Wascher. ``Minimum Wage Effects throughout the Distribution'', \href{https://doi.org/10.3368/jhr.XXXIX.2.425}{The Journal of Human Resources}.

\vspace{0.7em}

\noindent {\bf Own-wage elasticity estimate:} -0.227 (0.263)

\vspace{0.7em}

\noindent {\bf Source of estimate:} Weighted average

\subsection*{Neumark and Nizalova 2007}
\vspace{-0.7em}

\noindent David Neumark and Olena Nizalova. ``Minimum Wage Effects in the Longer Run'', \href{https://doi.org/10.3368/jhr.XLII.2.435}{The Journal of Human Resources}.

\vspace{0.7em}

\noindent {\bf Own-wage elasticity estimate:} -0.913 (0.579)

\vspace{0.7em}

\noindent {\bf Source of estimate:} Wage elasticity from Table 2, specification 1, panel "16-19". Employment semi-elasticity from Table 2, specification 2, panel "16-19", divided by 46.45, the mean employment from Table 1, specification "16-19-year olds", "Whole sample".

\subsection*{Neumark and Yen 2023}
\vspace{-0.7em}

\noindent David Neumark and Maysen Yen. ``The employment and redistributive effects of reducing or eliminating minimum wage tip credits'', \href{https://doi.org/10.1002/pam.22450}{Journal of Policy Analysis and Management}.

\vspace{0.7em}

\noindent {\bf Own-wage elasticity estimate:} -0.154 (0.41)

\vspace{0.7em}

\noindent {\bf Source of estimate:} Sum of the headline and tipped wage and employment elasticities to approximate the effect of a marginal increase in both the headline and tipped minimum wage. Wage and employment elasticities from Table 2, Panel A (1990-2019), for full-service restaurants (the sum of headline and tipped minimum wage coefficients for either limited service and full-limited specifications are not positive and statistically significant). Wage elasticity from specification 1 and employment elasticity from specification 4. adding the tipped and headline minimum effects together. Standard errors assume no covariance between the tipped and headline minimum wage effect estimates.

\subsection*{Orrenius and Zavodny 2008}
\vspace{-0.7em}

\noindent Pia M. Orrenius and Madeline Zavodny. ``The Effect of Minimum Wages on Immigrants' Employment and Earnings'', \href{https://doi.org/10.1177/001979390806100406}{ILR Review}.

\vspace{0.7em}

\noindent {\bf Own-wage elasticity estimate:} -0.4 (0.401)

\vspace{0.7em}

\noindent {\bf Source of estimate:} Wage elasticity from Table 3, row "Less-Educated Immigrants", specification 2. Employment elasticity from Table 4, panel A ("Real Minimum Wage"), row "Less-Educated Immigrants", specification 2.

\subsection*{Pereira 2003}
\vspace{-0.7em}

\noindent Sonia C. Pereira. ``The impact of minimum wages on youth employment in Portugal'', \href{https://doi.org/10.1016/S0014-2921(02)00209-X}{European Economic Review}.

\vspace{0.7em}

\noindent {\bf Own-wage elasticity estimate:} -2.31 (0.854)

\vspace{0.7em}

\noindent {\bf Source of estimate:} Employment effect from Table 2, column 2 (1986-1988), row 2 (ages 30-35), divided by mean employment of 0.785 referenced in footnote 16. Wage effect from Table 1, column 2 (1986 and 1988), row 2 (ages 30-35). Specifications chosen because of the reference to them on p. 236.

\subsection*{Rao and Risch 2024}
\vspace{-0.7em}

\noindent Nirupama Rao and Max Risch. ``Who's Afraid of the Minimum Wage? Measuring the Impacts on Independent Businesses Using Matched U.S. Tax Returns'', \href{https://www.nirupamarao.org/_files/ugd/ed3ee5_5b251b066aa74388917e8024285973db.pdf}{Working Paper}.

\vspace{0.7em}

\noindent {\bf Own-wage elasticity estimate:} -0.209 (0.011)

\vspace{0.7em}

\noindent {\bf Source of estimate:} Own-wage elasticity and standard error reported on p. 14.

\subsection*{Renkin Montialoux Siegenthaler 2022}
\vspace{-0.7em}

\noindent Tobias Renkin, Claire Montialoux and Michael Siegenthaler. ``The Pass-Through of Minimum Wages into U.S. Retail Prices: Evidence from Supermarket Scanner Data'', \href{https://doi.org/10.1162/rest_a_00981}{The Review of Economics and Statistics}.

\vspace{0.7em}

\noindent {\bf Own-wage elasticity estimate:} -0.093 (0.446)

\vspace{0.7em}

\noindent {\bf Source of estimate:} Wage and employment elasticities from Table 4, specification Grocery Stores, specification 1 ("Baseline"): wage elasticity from Panel A and employment elasticity from Panel B. The table also presents results for retail and restaurants but the focus of the paper is groceries.

\subsection*{Riley and Bondibene 2017}
\vspace{-0.7em}

\noindent Rebecca Riley and Chiara Rosazza Bondibene. ``Raising the standard: Minimum wages and firm productivity'', \href{http://dx.doi.org/10.1016/j.labeco.2016.11.010}{Labour Economics}.

\vspace{0.7em}

\noindent {\bf Own-wage elasticity estimate:} 0.064 (0.575)

\vspace{0.7em}

\noindent {\bf Source of estimate:} Wage and employment effects from Table 2, specification "OLS regression" and "£12,000" cutoff. Log change in wages from row "Labour costs", and log change in employment from row "Employment". The "£12,000" cutoff version is used because this is the version shown in the paper's figures.

\subsection*{Ruffini 2022}
\vspace{-0.7em}

\noindent Krista Ruffini. ``Worker Earnings, Service Quality, and Firm Profitability: Evidence from Nursing Homes and Minimum Wage Reforms'', \href{https://doi.org/10.1162/rest_a_01271}{The Review of Economics and Statistics}.

\vspace{0.7em}

\noindent {\bf Own-wage elasticity estimate:} 0.261 (0.202)

\vspace{0.7em}

\noindent {\bf Source of estimate:} Wage elasticity from Table 2, specification 3. Employment elasticity from Table 3, specification 5. Author communicated these preferred elasticities by email.

\subsection*{Sabia 2008}
\vspace{-0.7em}

\noindent Joseph J. Sabia. ``Minimum Wages and the Economic Well-Being of Single Mothers'', \href{https://doi.org/10.1002/pam.20379}{Journal of Policy Analysis and Management}.

\vspace{0.7em}

\noindent {\bf Own-wage elasticity estimate:} -0.888 (NA)

\vspace{0.7em}

\noindent {\bf Source of estimate:} Wage elasticity from Table 4, row "Single mothers with < high school education", column "Estimated Elasticity". Employment elasticity from Table 5, Panel II ("< HS Education"), row "Min. wage elasticity", specification 1 (Employment).

\subsection*{Sabia 2009a}
\vspace{-0.7em}

\noindent Joseph J. Sabia. ``Identifying Minimum Wage Effects: New Evidence from Monthly CPS Data'', \href{https://doi.org/10.1111/j.1468-232X.2009.00559.x}{Industrial Relations}.

\vspace{0.7em}

\noindent {\bf Own-wage elasticity estimate:} -1.874 (0.517)

\vspace{0.7em}

\noindent {\bf Source of estimate:} Wage elasticity from Table 3, row "MINWAGE", specification 6 (1979-2004, Year effects). Employment elasticity from Table 4, row "Min wage elasticity", specification 6 (1979-2004, Year effects), with standard error from semi-elasticity scaled by the ratio of the elastiticy to semi-elasticity.

\subsection*{Sabia 2009b}
\vspace{-0.7em}

\noindent Joseph J. Sabia. ``The Effects of Minimum Wage Increases on Retail Employment and Hours: New Evidence from Monthly CPS Data'', \href{https://doi.org/10.1007/s12122-008-9054-1}{Journal of Labor Research}.

\vspace{0.7em}

\noindent {\bf Own-wage elasticity estimate:} -0.592 (0.266)

\vspace{0.7em}

\noindent {\bf Source of estimate:} Wage elasticity from Table 2, specfication 1, row "Min wage elasticity". Employment elasticity from Table 3, specification 1, row "Min wage elasticity". Standard errors are calculated from multiplying the elasticity estimate by the the ratio of the semielasticity standard error to semielasticity point estimate.

\subsection*{Sabia Burkhauser Hansen 2012}
\vspace{-0.7em}

\noindent Joseph J. Sabia, Richard V. Burkhauser and Benjamin Hansen. ``Are the Effects of Minimum Wage Increases Always Small? New Evidence from a Case Study of New York State'', \href{https://doi.org/10.1177/001979391206500207}{ILR Review}.

\vspace{0.7em}

\noindent {\bf Own-wage elasticity estimate:} -2.213 (1.275)

\vspace{0.7em}

\noindent {\bf Source of estimate:} wage: Table 2 column 5; emp: Table 3 column 5

\subsection*{Slichter 2023}
\vspace{-0.7em}

\noindent David Slichter. ``The employment effects of the minimum wage: A selection ratio approach to measuring treatment effects'', \href{https://doi.org/10.1002/jae.2954}{Journal of Applied Econometrics}.

\vspace{0.7em}

\noindent {\bf Own-wage elasticity estimate:} -0.174 (0.182)

\vspace{0.7em}

\noindent {\bf Source of estimate:} Simple average of earnings and employment changes across five time period specifications, "t" through "t+4". Earnings effects from Appendix Table 6 and employment effects from Panel A of Table 2.

\subsection*{Stewart 2004}
\vspace{-0.7em}

\noindent Mark B. Stewart. ``The Impact of the Introduction of the U.K. Minimum Wage on the Employment Probabilities of Low-Wage Workers'', \href{https://doi.org/10.1162/154247604323015481}{Journal of the European Economic Association}.

\vspace{0.7em}

\noindent {\bf Own-wage elasticity estimate:} 0.262 (0.516)

\vspace{0.7em}

\noindent {\bf Source of estimate:} Wage elasticity from Table 1, LFS actual hours, Full set of time dummies added. Employment elasticity from dividing the retentation rate percentage point change by an estimate of the average retention rate. Percentage point change in the retention rate is the simple average of Adult men and Adult women Table 2 raw linear difference-in-difference estimates, wage based on actual hours. Average retention rate of 81.4% estimated from dividing the total wage sample size in notes of Table 1 (44076) by the total employment sample size in notes of Table 2 (54165).

\subsection*{Thompson 2009}
\vspace{-0.7em}

\noindent Jeffrey P. Thompson. ``Using Local Labor Market Data to Re-examine the Employment Effects of the Minimum Wage'', \href{https://doi.org/10.1177/001979390906200305}{ILR Review}.

\vspace{0.7em}

\noindent {\bf Own-wage elasticity estimate:} -0.392 (0.161)

\vspace{0.7em}

\noindent {\bf Source of estimate:} Wage: Table 6, column 1 row 1; Emp: Table 5, column 1 row 1

\subsection*{Totty 2017}
\vspace{-0.7em}

\noindent Evan Totty. ``The Effect of Minimum Wages on Employment: A Factor Model Approach'', \href{https://doi.org/10.1111/ecin.12472}{Economic Inquiry}.

\vspace{0.7em}

\noindent {\bf Own-wage elasticity estimate:} -0.155 (0.109)

\vspace{0.7em}

\noindent {\bf Source of estimate:} Wage: Table 9, IFE; Emp: Table 3, IFE

\subsection*{Vadean and Allen 2021}
\vspace{-0.7em}

\noindent Florin Vadean and Stephen Allen. ``The Effects of Minimum Wage Policy on the Long-Term Care Sector in England'', \href{https://doi.org/10.1111/bjir.12572}{British Journal of Industrial Relations}.

\vspace{0.7em}

\noindent {\bf Own-wage elasticity estimate:} -0.036 (0.402)

\vspace{0.7em}

\noindent {\bf Source of estimate:} Weighted average of the employment and earnings elasticities for residential care (Table 3) and domiciliary care (Table 4). Weights are employment shares from p.311. Wage effects from column 1, row 1, and employment effects from column 3, row 1 of each table.

\subsection*{Vergara 2023}
\vspace{-0.7em}

\noindent Damián Vergara. ``Minimum Wages and Optimal Redistribution'', \href{https://dvergarad.github.io/files/JMP_DV.pdf}{Working Paper}.

\vspace{0.7em}

\noindent {\bf Own-wage elasticity estimate:} 0.448 (NA)

\vspace{0.7em}

\noindent {\bf Source of estimate:} Elasticities from row "Second stage (elasticity)" of Table A.3 using the year X Census division specification because that is used in the simulations. Wage elasticity from column 3 and employment elasticity from column 6.

\subsection*{Wiltshire McPherson Reich Sosinskiy 2024}
\vspace{-0.7em}

\noindent Justin C. Wiltshire, Carl McPherson, Michael Reich and Denis Sosinskiy. ``Minimum Wage Effects and Monopsony Explanations'', \href{https://irle.berkeley.edu/wp-content/uploads/2023/09/minimum-wage-effects-and-monopsony-explanations-augustREVISED.pdf}{IRLE Working Paper}.

\vspace{0.7em}

\noindent {\bf Own-wage elasticity estimate:} 1.55 (0.584)

\vspace{0.7em}

\noindent {\bf Source of estimate:} Own-wage elasticity reported in Table 4, Panel B, which is the authors' preferred estimate via email correspondence.

\subsection*{Wursten 2020}
\vspace{-0.7em}

\noindent Jesse Wursten. ``Is Politics the Missing Piece of the Minimum Wage Puzzle?'', \href{http://dx.doi.org/10.13140/RG.2.2.14177.20329}{Working Paper}.

\vspace{0.7em}

\noindent {\bf Own-wage elasticity estimate:} -0.221 (0.15)

\vspace{0.7em}

\noindent {\bf Source of estimate:} wage = Table 5, panel A, column 5; emp = table 2, panel B, column 1

\subsection*{Wursten 2021}
\vspace{-0.7em}

\noindent Jesse Wursten. ``Estimating the earnings and employment effects of the minimum wage through differences in exposure across US counties'', \href{https://www.dropbox.com/s/2f5j0mjyama8q0p/CAMWE_draft36.pdf?dl=0}{Working Paper}.

\vspace{0.7em}

\noindent {\bf Own-wage elasticity estimate:} -0.269 (0.163)

\vspace{0.7em}

\noindent {\bf Source of estimate:} Simple average of the elasticities for low and high vulnerability counties on p.12. Elasticity estimates (standard errors) for earnings are 0.06 (0.02) and 0.20 (0.03) and for employment are -0.02 (0.02) and -0.05 (0.02). Estimates and standard errors provided from author by email.

\subsection*{Wursten and Reich 2023a}
\vspace{-0.7em}

\noindent Jesse Wursten and Michael Reich. ``Racial inequality in frictional labor markets: Evidence from minimum wages'', \href{https://doi.org/10.1016/j.labeco.2023.102344}{Labour Economics}.

\vspace{0.7em}

\noindent {\bf Own-wage elasticity estimate:} 0.003 (0.019)

\vspace{0.7em}

\noindent {\bf Source of estimate:} Weighted average of HSOL < \$20 OWEs in Table E4 using shares in Table 2. OWE estimates (and standard errors) are Black 0.04 (0.04); White 0.00 (0.01); and Hispanic -0.01 (0.03). Sample shares are Black 0.11, White 0.71, and Hispanic 0.12.

\subsection*{Wursten and Reich 2023b}
\vspace{-0.7em}

\noindent Jesse Wursten and Michael Reich. ``Small Business and the Minimum Wage'', \href{https://irle.berkeley.edu/wp-content/uploads/2023/03/Small-Businesses-and-the-Minimum-Wage-3-14-23.pdf}{IRLE Working Paper}.

\vspace{0.7em}

\noindent {\bf Own-wage elasticity estimate:} -0.241 (0.182)

\vspace{0.7em}

\noindent {\bf Source of estimate:} Weighted average of three wage and employment elasticities for restaurants, teens, and young adults, where the weights are the "Group size in 2019Q4": Restaurants, Table 3, Firm size All; Teens 14-18, Table 5, Firm size All; Young adults 19-21, Table 6, Firm Size All.

\subsection*{Zavodny 2000}
\vspace{-0.7em}

\noindent Madeline Zavodny. ``The Effect of the Minimum Wage on Employment and Hours'', \href{https://doi.org/10.1016/S0927-5371(00)00021-X}{Labour Economics}.

\vspace{0.7em}

\noindent {\bf Own-wage elasticity estimate:} -0.174 (0.896)

\vspace{0.7em}

\noindent {\bf Source of estimate:} Employment elasticity from Table 1, specification 1, row "Log of real minimum wage". Hourly wage semi-elasticity estimate and standard error from Table 1, specification 7, row "Log of real minimum wage", divided by 5.68, reported in Table 2 for the Total sample as the "Average real hourly wage in the first year".

